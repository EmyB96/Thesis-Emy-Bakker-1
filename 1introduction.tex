\section{Introduction} \label{sec:intro}
% \begin{itemize}
% \item Bevat je onderzoeksvraag (of vragen)
% \item Plaatst je vraag in de bestaande literatuur.
% \end{itemize}

% Je onderzoeksvraag is leidend voor je hele scriptie. Alles wat je doet moet uiteindelijk terug te voeren zijn op 1 doel: het beantwoorden van die vraag. 

% Typisch zal je het dan ook zo doen:

% Mijn onderzoeksvraag is onderverdeeld in de volgende deelvragen:

% \begin{description}
% \item[RQ1] \ldots We   beantwoorden deze vraag  door het volgende te doen/ antwoord op de volgende vragen te vinden/ \ldots
% \begin{enumerate}
% \item Vragen op dit niveau kan je echt beantwoorden, en dat doe je in je Evaluatie sectie~\ref{sec:eva}.
% \end{enumerate}
% \item[RQ2] \ldots
% \item[RQ3] \ldots
% \end{description}
% %
% Je Evaluatie sectie~\ref{sec:eva} bevat evenveel subsecties als je deelvragen hebt. En in elke sectie beantwoord je dan die deelvraag met behulp van de vragen op het onderste niveau.

% In je conclusies kan je dan je hoofdvraag gaan beantwoorden op basis van al het eerder vergaarde bewijs.

A few days ago I saw a 'meme' on my Instagram timeline which stated: "Polarization cannot be avoided by denying there is a problem". My immediate thought was: "But how can it be avoided or at least, what can be of impact?!".

As we are increasingly experiencing nowadays, online social networks are an important place for the dissemination of opinions. Dissemination is the spread of for example ideas, information or news and it can result in consensus (everyone has the same opinion) or polarization (the different opinions are divided) due to structures in such a network. Structures in a social network include the interaction between different (hierarchical) social circles. Hierarchical social circles can, for example, consist of individuals who participate in university, faculty, department, and cohort-based circles, which all represent different levels of interaction with different degrees of cohesiveness.

The question that results from this, is which level of interaction matters the most to the process of opinion distribution. Once this is known, opinion dynamics in online social networks will become more comprehensible and will enable the prospects of using such processes to automatically discover social circles.

Data extracted from online social media with well-defined social circles can help to find an answer to the formulated research question:

\centerline{\textit{To what extent can social circles impact opinion dynamics?}}

\noindent The following sub-questions arise from this:
\begin{enumerate}
    \item How can the previously identified social circles be characterized and visualized?
    \item What is the impact of social circles on the final distribution of opinions, particularly in the final levels of consensus and polarization?
    \item What is the correlation between the final distribution of opinions and social circles?
\end{enumerate}


\paragraph{Overview of thesis}
% Hier geef je even kort weer wat in elke sectie staat.